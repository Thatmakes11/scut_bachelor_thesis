\documentclass[UTF8, twoside, 12pt]{ctexart}

% Packages

\usepackage{graphicx, fancyhdr, subfigure, enumitem}
\usepackage[bottom=3cm]{geometry}
\usepackage{setspace, parskip}
\usepackage{fontspec, xeCJK}
\usepackage{titlesec, titletoc}
\usepackage[subfigure]{tocloft}
\usepackage{amsmath}
\usepackage{caption, tabularx, array, etoolbox}
\usepackage[backend=biber, style=numeric, maxnames=3, minnames=3]{biblatex}
\addbibresource{contents/references.bib}


% document class settings

\geometry{
  left=72pt,
  right=72pt,
  bindingoffset=0cm
}

\setcounter{secnumdepth}{4}

\titleformat{\paragraph}[block]{\zihao{-4}\heiti}{\theparagraph}{1em}{}
\titlespacing*{\paragraph}{0pt}{\baselineskip}{0pt}

\renewcommand\theparagraph{\thesubsubsection.\arabic{paragraph}}


% reference settings
\AtEveryCite{\fontsize{12}{14}\selectfont}

\DeclareCiteCommand{\supercite}[\mkbibsuperscript]
  {\iffieldundef{prenote}
     {}
     {\BibliographyWarning{Ignoring prenote argument}}%
   \iffieldundef{postnote}
     {}
     {\BibliographyWarning{Ignoring postnote argument}}%
   \bibopenbracket}%
  {\usebibmacro{citeindex}%
   \usebibmacro{cite}}
  {\supercitedelim}
  {\bibclosebracket}


% table of contents settings
\renewcommand{\cftsecleader}{\cftdotfill{\cftdotsep}}
\renewcommand{\cftsecfont}{\heiti}
\renewcommand{\cftsecpagefont}{\songti}
\renewcommand{\contentsname}{\hfill \heiti 目录 \hfill}


% title settings
\titleformat{\section}[block]
{\centering\fontsize{18pt}{16pt}\heiti}
{第\chinese{section}章\hspace{1em}}
{0pt}
{\vspace{0.5\baselineskip}\hspace{-0.5em}}
\titlespacing{\section}{0pt}{0.5\baselineskip}{0.5\baselineskip}

\renewcommand{\sectionmark}[1]{\markboth{第\chinese{section}章\hspace{0.5em}#1}{}}

\ctexset {
    section = {
        name = {第,章},
        number = \chinese{section},
    }
}


% subsection settings
\titleformat{\subsection}
{\bfseries\fontsize{15pt}{12pt}\heiti}
{\normalfont\thesubsection}{0.5em}{}
\titlespacing{\subsection}{0pt}{0.5\baselineskip}{0.5\baselineskip}


% subsubsection settings
\titleformat{\subsubsection}
{\bfseries\fontsize{14pt}{12pt}\heiti}
{\normalfont\thesubsubsection}{0.5em}{}


% paragraph indent settings
\setlength{\parindent}{2em}


% font settings
\setmainfont{Times New Roman}
\setCJKmainfont{SimSun}

% if you are using a macOS 14:
% \setCJKmainfont{STSongti-SC-Light} 


% page head and foot settings
\pagestyle{fancy}
\fancyhf{}
\fancyfoot[C]{\thepage}

\fancyhead[CO]{\leftmark}
\fancyhead[CE]{华南理工大学学士学位论文}

\renewcommand{\headrulewidth}{0.4pt}

\setlength{\headheight}{14.5pt}


% equation settings
\numberwithin{equation}{section}


% table settings
\renewcommand{\thetable}{\thesection.\arabic{table}}
\renewcommand{\arraystretch}{2}
\tabcolsep=1cm

\DeclareCaptionFont{tablefontsize}{\fontsize{10.5pt}{12.6pt}\selectfont}
\AtBeginEnvironment{tabular}{\fontsize{10.5pt}{12.6pt}\selectfont}
\captionsetup[table]{font={tablefontsize}}
\captionsetup[table]{labelsep=space}


% figure settings
\renewcommand{\thefigure}{\thesection.\arabic{figure}}
\captionsetup[figure]{font={tablefontsize}}
\captionsetup[figure]{labelsep=space}
\renewcommand\thesubfigure{\alph{subfigure})}



\begin{document}

\title{基于卷积神经网络的手写数字及写字人识别}
\author{Sun Chang}
\date{\today}

\newcommand{\authorname}{Your name}
\newcommand{\school}{Your school}
\newcommand{\major}{Your major}
\newcommand{\studentid}{Your student id}
\newcommand{\tutor}{Your tutor}
\newcommand{\submitdate}{2024年\ 05月\ 00日}

\thispagestyle{empty}

\begin{figure}[!t]
    \centering
    \includegraphics[width=.8\textwidth]{figs/logo.jpg}
    \label{fig:logo}
\end{figure}

{\heiti \zihao{0} \centerline{\ 本科毕业设计(论文)}}

\makeatletter
\renewcommand{\maketitle}{
    \begin{center}
        \heiti \zihao{2}\bfseries{\@title}
    \end{center}
}
\makeatother

\vfill
\maketitle
\vfill

\begin{table}[!b]
    \centering
    \begin{spacing}{1.6}
        \tabcolsep=0.5cm
        \begin{tabular}{>{\zihao{-3}\heiti}l>{\zihao{-3}\heiti}c}
            \makebox[4em]{学\qquad 院} & \underline{\makebox[10em][c]{\school}} \\
            \makebox[4em]{专\qquad 业} & \underline{\makebox[10em][c]{\major}} \\
            \makebox[4em]{学生姓名} & \underline{\makebox[10em][c]{\authorname}} \\
            \makebox[4em]{学生学号} & \underline{\makebox[10em][c]{\studentid}} \\
            \makebox[4em]{指导教师} & \underline{\makebox[10em][c]{\tutor}} \\
            \makebox[4em]{完成日期} & \underline{\makebox[10em][c]{\submitdate}} \\
        \end{tabular}
    \end{spacing}
\end{table}

\newpage

\thispagestyle{empty}

\begin{center}
    \vspace{0.5\baselineskip}
    {\heiti \zihao{2} 华南理工大学} \\
    \vspace{0.5\baselineskip}
    {\heiti \zihao{2} 学位论文原创性声明}
    \vspace{0.5\baselineskip}
\end{center}

\begin{spacing}{1.7}
    \songti \zihao{4}
    本人郑重声明:所呈交的论文是本人在导师的指导下独立进行研究所取得的研究成果。
    除了文中特别加以标注引用的内容外,本论文不包含任何其他个人或集体已经发表或撰写的成果作品。
    对本文的研究做出重要贡献的个人和集体,均已在文中以明确方式标明。本人完全意识到本声明的法律后果由本人承担。 
    \begin{center}
        \makebox[13em][l]{作者签名:} \makebox[13em]{日期:\quad 年\quad 月\quad 日} \\
    \end{center}
\end{spacing}

\begin{center}
    \vspace{0.5\baselineskip}
    {\heiti \zihao{2} 学位论文版权使用授权书} \\
    \vspace{0.5\baselineskip}
\end{center}

\begin{spacing}{1.7}
    \songti \zihao{4}
    本学位论文作者完全了解学校有关保留、使用学位论文的规定,即:学校有权保存并向国家有关部门或机构送交论文的复印件和电子版,
    允许学位论文被查阅;学校可以公布学位论文的全部或部分内容,可以允许采用影印、缩印或其它复制手段保存、汇编学位论文。本人电子文档的内容和纸质论文的内容相一致。 
    \begin{center}
        \makebox[13em][l]{作者签名:} \makebox[13em][l]{日期:\quad 年\quad 月\quad 日} \\
        \makebox[13em][l]{指导教师签名:} \makebox[13em][l]{日期:\quad 年\quad 月\quad 日} \\
        \makebox[13em][l]{作者联系电话:} \makebox[13em][l]{电子邮箱:} \\
    \end{center}
\end{spacing}

\newpage

\thispagestyle{plain}
\pagenumbering{Roman}

\addcontentsline{toc}{section}{摘要}

\begin{center}
    \vspace{0.5\baselineskip}
    {\heiti \zihao{-2} 摘要}
    \vspace{0.5\baselineskip}
\end{center}

\setstretch{1.7}
\songti \zihao{-4}
摘要正文共400—600个字;小四号,宋体,1.5倍行距,段首行空两个汉字\\
\hspace*{\parindent}炔烃和叠氮化合物的点击化学反应,有着快速、百分百原子利用率、产物高选择性等众多优点,被誉为点击化学中的精华。基于此反应拓展而来的点击聚合反应,迅速在高分子材料领域获得了了广泛关注和应用。\\
\hspace*{\parindent}……\\
\hspace*{\parindent}我们还尝试了采用不同单体,在最优条件下进行反应,均获得了高分子产物。表明了该反应体系的普适性。
\vspace{\baselineskip}

\noindent\textbf{\heiti \zihao{-4} 关键词:} 
\songti \zihao{-4} 多变量系统;预测控制;环境试验设备

\newpage

\thispagestyle{plain}

\addcontentsline{toc}{section}{Abstract}

\begin{center}
    \vspace{0.5\baselineskip}
    {\zihao{-2} Abstract}
    \vspace{0.5\baselineskip}
\end{center}

\setstretch{1.7}
\zihao{-4}
Artificial Neuron Network (ANN) simulates human being’s brain function and build the network structure. Convolutional Neural Network (CNN) have many advantage, such as ……\\
\hspace*{\parindent}This paper introduces the common pretreatment method of image, such as collecting image, normalization, graying and binarization. And apply these to the handwritten numeral recognition experiment and handwritten numerals writer recognition experiments.

\vspace{\baselineskip}

\noindent\textbf{\heiti \zihao{-4} Keywords: } 
\zihao{-4} Writer recognition;Convolutional Neural Network;Handwritten character recognition

\newpage

\thispagestyle{empty}

\clearpage
\addcontentsline{toc}{section}{目录}

\tableofcontents

\newpage


\input{contents/body/chap1_introduction.tex}
\section{卷积神经网络的基础知识}

\subsection{卷积神经网络的网络结构}
卷积神经网络作为深度学习的一个分支,在网络结构上同样含有深度学习的“深度”性。
网络拓扑结构是一个多层的神经网络[8],网络的每一层由多个独立的神经元组成的二维平面组成。
网络一般分为输入层、卷积层、池化层、全连接层、输出层等。

\subsubsection{输入层}
因为卷积神经网络可以直接的接受二维的视觉模式[9],所以我们可以直接把简单预处理后的二维图像输入到输入层中。

\subsubsection{输出}
......

\subsection{卷积神经网络的学习规律}
......

\subsubsection{前向传播}
如果用l来表示当前的网络层,那么当前网络层的输出如公式(2-1)所示:
\begin{equation}
    x^l = f(u^l), \mbox{其中} u^l = w^l \cdot x^{l-1} + b^l
\end{equation}
其中$f(.)$为网络的输出激活函数。在本文实验中,网络的输出激活函数选用sigmoid函数,因此网络的输出均值一般来说趋于0。

\subsubsection{反向传播}
......

\subsubsection{学习特征图的组合}
......

\subsection{本章小结}
......

\newpage

\section{基于卷积神经的手写数字及写字人识别算法设计}

\subsection{输入输出层的设计}
......

\subsection{隐藏层的设计}
......

\subsection{本章小结}
......

\newpage

\section{手写数字及写字人识别实验过程及其结果}

\subsection{手写数字识别实验}

\subsubsection{样本简介}
本论文的手写数字识别实验当中所用的样本分为两类,一类是训练样本集,另一类是测试样本集。

实验当中的训练样本集采用的是手写数字MNIST数据库。这个数据库当中包含训练集样本60000个样例和测试集样本10000个样例。
MNIST数据库当中的数字样本已经全部大小归一化灰度化并且集中到同一个固定大小的图像当中。
该数据库包括MST的SD-1和SD-3数据库,当中包含一系列的二级制的手写数字图像。
其中SD-1的收集者来源是某高中的在校学生,而SD-3是由人口调查局员工收集的。
则我们的训练样本集也就是MNIST当中的训练样本集有30000个样本来自SD-3,而另外30000个样本来自SD-1。这60000个训练样本分别来自约250个采集者。

\subsubsection{Writer Depend类数字识别实验}
......

\paragraph{ABCvsA数字识别实验}
实验内容:以A写字人、B写字人和C写字人,合计3000个数字0到9的数字图像数据为训练样本集。
A写字人的1000个数字0到9的数字图像数据为测试样本集。学习率为1,单次训练样本数为10个,共训练40次。
若识别所得数字与给定的标签匹配,则视为正确;不匹配则视为错误。
\begin{table}[!htbp]
    \centering
    \caption{ABCvsA数字识别实验结果}
    \begin{tabular}[\textwidth]{c|c|c|c}
        \hline
        训练样本 & ABC & 样本个数 & 3000 \\
        \hline
        测试样本 & A & 样本个数 & 1000 \\
        \hline
        训练样本 & $\backslash$ & 单次训练样本数 & 10 \\
        \hline
        学习率 & 1 & 正确率 & 99.50\% \\
        \hline
    \end{tabular}
\end{table}

\paragraph{ABCvsABC数字识别实验}
实验内容:以A写字人、B写字人和C写字人,合计3000个数字0到9的数字图像数据为总样本集。
在总样本集当中随机抽取2400个为训练样本集,余下的600个为测试样本集。
学习率为1,单次训练样本数为10个,共训练40次。若识别所得数字与给定的标签匹配,则视为正确;不匹配则视为错误。
\begin{table}[!htbp]
    \centering
    \caption{ABCvsABC数字识别实验结果}
    \begin{tabular}[\textwidth]{c|c|c|c}
        \hline
        训练样本 & ABC & 样本个数 & 2400 \\
        \hline
        测试样本 & ABC & 样本个数 & 600 \\
        \hline
        训练样本 & 40 & 单次训练样本数 & 10 \\
        \hline
        学习率 & 1 & 正确率 & 92.00\% \\
        \hline
    \end{tabular}
\end{table}

\subsubsection{Writer Depend类数字识别实验结果分析}
下面我们选取Writer Depend类数字识别实验当中的两个典型的例子ABCvsA数字识别实验以及MNIST\& ABCvsA数字识别实验的结果做详细分析。
我们从ABCvsA数字识别实验中的训练样本集和测试样本集的手写数字图像样本集当中分别随机抽取一幅图像如图4-1所示。

下面我们对上述的训练集和测试集进行40次学习率为2,单次训练样本为10的迭代,得到错误率为0.50\%,而其中每次训练时的误差值组成的历史误差值画图分析如下:

......

\begin{figure}[!htbp]
    \centering
    \subfigure[实验训练集]{
      \includegraphics[width=0.45\textwidth]{figs/train.png}
      \label{fig:train}
    }
    \subfigure[实验测试集]{
      \includegraphics[width=0.45\textwidth]{figs/test.png}
      \label{fig:test}
    }
    \caption{ABCvsA数字识别实验集}
    \label{fig:both_images}
\end{figure}

\subsubsection{Writer Independ类数字识别实验}
实验内容:以MNIST数据库为训练样本集,共计60000个训练样本。以A写字人合计1000个数字0到9的数字图像数据为测试样本集写字人识别实验

......

\subsubsection{样本简介}
......

\subsubsection{两位写字人识别实验}
\paragraph{单个数字的写字人识别实验}
实验内容:以A写字人,合计800个数字5的数字图像数据加上B写字人,合计800个数字5的数字图像数据,
共计1600个样本为总样本集。随机选取其中的1200个样本为训练样本集,其余的400个样本为测试样本集。学习率为2,
单次训练样本数为10个,共训练30次。若识别所得写字人与给定的标签匹配,则视为正确;不匹配则视为错误。
\begin{table}[!htbp]
    \centering
    \caption{单个数字写字人识别实验结果}
    \begin{tabular}[\textwidth]{c|c|c|c}
        \hline
        训练样本 & A5\&B5 & 样本个数 & 1200 \\
        \hline
        测试样本 & A5\&B5 & 样本个数 & 400 \\
        \hline
        训练样本 & 30 & 单次训练样本数 & 10 \\
        \hline
        学习率 & 2 & 正确率 & 99.75\% \\
        \hline
    \end{tabular}
\end{table}

\paragraph{单个数字的写字人识别实验结果分析}
......

\subsection{本章小结}
......

\newpage

\section*{结论}
\addcontentsline{toc}{section}{结论}

{
\renewcommand{\thesubsection}{\arabic{subsection}.}
\setcounter{subsection}{0}
\subsection{论文工作总结}

\subsection{工作展望}
}


\newpage


\printbibliography[title={参考文献}]
\addcontentsline{toc}{section}{参考文献}
\newpage

\section*{致谢}
\addcontentsline{toc}{section}{致谢}

\thispagestyle{plain}
感谢......


\end{document}
